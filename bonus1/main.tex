\documentclass[12pt]{article}
\usepackage[a4paper,margin=1in]{geometry}
\usepackage{amsmath,amssymb}
\usepackage{graphicx}
\usepackage{siunitx}
\sisetup{per-mode=symbol}


\title{Bonus Question}
\author{ai25btech11015 -- M Sai Rithik}
\date{}
\begin{document}
\maketitle


\section*{Q : Prove the Condition for Two Points to Lie on the Same or Opposite Side of a Line}

Let the line be 
\[
L: \; l_1x + l_2y + c = 0
\]
and let two points be 
\[
P_1(x_1, y_1), \quad P_2(x_2, y_2).
\]



\subsection*{Using Section Formula}
If a point $P(x,y)$ divides $P_1P_2$ internally in the ratio $m:1$, then
\[
x = \frac{mx_2 + x_1}{m+1}, \qquad 
y = \frac{my_2 + y_1}{m+1}.
\]
If m is > 0 then points lie on opposite side 
If m is < o then points lie on same side 



\subsection*{Substituting in Line Equation}
Substituting $(x,y)$ into the line equation:
\[
l_1 \left( \frac{mx_2 + x_1}{m+1} \right) 
+ l_2 \left( \frac{my_2 + y_1}{m+1} \right) + c = 0
\]

Multiplying through by $(m+1)$:
\[
m(l_1x_2 + l_2y_2 + c) + (l_1x_1 + l_2y_1 + c) = 0
\]

\[
mL_2 + L_1 = 0
\]

where
\[
L_1 = l_1x_1 + l_2y_1 + c, \qquad
L_2 = l_1x_2 + l_2y_2 + c.
\]

Thus,
\[
m = -\frac{L_1}{L_2}.
\]



- If $m > 0 \;\;\Rightarrow\;\; L_1L_2 < 0$, the point lies \textbf{between} $P_1$ and $P_2$ $\implies$ points are on \textbf{opposite sides} of the line.  
- If $m < 0 \;\;\Rightarrow\;\; L_1L_2 > 0$, the section ratio is negative $\implies$ points are on the \textbf{same side} of the line.


\subsection*{Hence}
\[
L_1 \cdot L_2 > 0 \;\; \Rightarrow \;\; \text{Same side of the line.}
\]
\[
L_1 \cdot L_2 < 0 \;\; \Rightarrow \;\; \text{Opposite sides of the line.}
\]

\end{document}

\documentclass[12pt]{article}
\usepackage[a4paper,margin=1in]{geometry}
\usepackage{amsmath,amssymb}
\usepackage{graphicx}
\usepackage{siunitx}
\sisetup{per-mode=symbol}


\title{Matrix 1.7.1}
\author{ai25btech11015 -- M Sai Rithik}
\date{}

\begin{document}
\maketitle

\section*{Question}
Show that the points \((0,0)\), \((2m,-4)\), and \((3,6)\) are collinear, and hence find \(m\), using the rank method.

\section*{Solution}

Let the given points be
\[
A = (0,0), \quad B = (2m,-4), \quad C = (3,6).
\]

\subsection*{Step 1: Form vectors}
\[
AB = B - A = \begin{bmatrix} 2m \\ -4 \end{bmatrix}, 
\quad AC = C - A = \begin{bmatrix} 3 \\ 6 \end{bmatrix}.
\]

\subsection*{Step 2: Matrix form}
Construct the matrix
\[
M = \begin{bmatrix}
2m & 3 \\
-4 & 6
\end{bmatrix}.
\]

For the points to be collinear, the two vectors \(AB\) and \(AC\) must be linearly dependent.  
This means
\[
\text{rank}(M) = 1 \quad \Leftrightarrow \quad \det(M) = 0.
\]

\[
M=\begin{bmatrix}
2m & 3\\
-4 & 6
\end{bmatrix}.
\]

We use RREF \(M\) and look for when its rank drops below \(2\).

\[
\begin{aligned}
\begin{bmatrix}
2m & 3\\
-4 & 6
\end{bmatrix}
&\xrightarrow{\;R_1 \leftrightarrow R_2\;}
\begin{bmatrix}
-4 & 6\\
2m & 3
\end{bmatrix}
\xrightarrow{\;R_1\gets -\tfrac14 R_1\;}
\begin{bmatrix}
1 & -\tfrac{3}{2}\\
2m & 3
\end{bmatrix}\\[6pt]
&\xrightarrow{\;R_2\gets R_2-2m\,R_1\;}
\begin{bmatrix}
1 & -\tfrac{3}{2}\\
0 & 3(m+1)
\end{bmatrix}.
\end{aligned}
\]

If \(m\neq -1\), the second row has a pivot (divide by \(3(m+1)\)), so the RREF is \(I_2\) and \(\operatorname{rank}(M)=2\).
For the rank to drop (and hence the RREF to have a zero row), we need
\[
3(m+1)=0 \;\;\Rightarrow\;\; m=-1.
\]

When \(m=-1\),
\[
\begin{bmatrix}
1 & -\tfrac{3}{2}\\
0 & 0
\end{bmatrix}
\]
is the reduced row-echelon form (rank \(=1\)).

\medskip

\section*{Final Answer}
The given points are collinear when
\[
\boxed{m = -1}
\]

\begin{center}
\end{center}

\begin{figure}[h!]
    \centering
    \includegraphics[width=0.65\linewidth]{figs/fig.png}
    \caption{Graph}
\end{figure}

\end{document}

\documentclass{beamer}
\mode<presentation>
\usepackage{amsmath}
\usepackage{amssymb}
\usepackage{adjustbox}
\usepackage{subcaption}
\usepackage{enumitem}
\usepackage{multicol}
\usepackage{mathtools}
\usepackage{listings}
\usepackage{url}
\def\UrlBreaks{\do\/\do-}
\usetheme{Boadilla}
\usecolortheme{lily}

\setbeamertemplate{footline}
{
  \leavevmode%
  \hbox{%
  \begin{beamercolorbox}[wd=\paperwidth,ht=2.25ex,dp=1ex,right]{author in head/foot}%
    \insertframenumber{} / \inserttotalframenumber\hspace*{2ex} 
  \end{beamercolorbox}}%
  \vskip0pt%
}
\setbeamertemplate{navigation symbols}{}

\title{1.7.1 -- Matgeo Assignment}
\author{ai25btech11015 -- M Sai Rithik}
\date{}

\begin{document}
\frame{\titlepage}
\begin{frame}{Question}
Show that the points \((0,0)\), \((2m,-4)\), and \((3,6)\) are collinear, and hence find \(m\), using the rank method.
\end{frame}

\begin{frame}{Step 1: Form vectors}
\[
A = (0,0), \quad B = (2m,-4), \quad C = (3,6)
\]
\[
AB = \begin{bmatrix}2m \\ -4\end{bmatrix}, 
\quad AC = \begin{bmatrix}3 \\ 6\end{bmatrix}
\]
\end{frame}

\begin{frame}{Step 2: Matrix form}
Form the matrix with \(AB\) and \(AC\) as columns:
\[
M = \begin{bmatrix}
2m & 3 \\
-4 & 6
\end{bmatrix}
\]

For collinearity, \(\text{rank}(M) = 1\), i.e. \(\det(M) = 0\).
\end{frame}

\begin{frame}{Step 3: Determinant}
\[
\det(M) = (2m)(6) - (-4)(3)
\]
\[
\det(M) = 12m + 12
\]

For collinearity:
\[
12m + 12 = 0 \quad \Rightarrow \quad m = -1
\]
\end{frame}

\begin{frame}{Final Answer}
The points \((0,0)\), \((2m,-4)\), and \((3,6)\) are collinear when
\[
\boxed{m = -1}
\]
\end{frame}

\end{document}
